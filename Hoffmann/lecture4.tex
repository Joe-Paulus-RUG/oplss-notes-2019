\documentclass{article}
\usepackage[utf8]{inputenc}
\usepackage[hyphens]{url} 
\usepackage[breaklinks]{hyperref}
\usepackage{amsmath,amsthm,stmaryrd,amssymb}
\usepackage{bussproofs}
\usepackage{listings}
\usepackage{tikz}
\usepackage{rotating}
\usetikzlibrary{matrix,patterns}

\newtheorem{theorem}{Theorem}[section]
\newtheorem{definition}{Definition}[theorem]
\newtheorem{example}{Example}

\newcommand{\ldash}[2]{\vdash^{\hspace{-.4em}#1}_{\hspace{-.4em}#2}}

\title{Resource Analysis \\ \Large{Lecture 4}} 

\begin{document}

\maketitle
\section{Recap: Soundness}

\begin{theorem}[Progress]
If $\ldash{q}{q'} e : \tau $ and $p \geq q$ then either $e$ is a value or $\exists e',p'$ s.t. $\langle e,p \rangle \mapsto \langle e',p'\rangle$.
\end{theorem}

\begin{theorem}[Preservation]
If $\ldash{q}{q'} e : \tau$ , $p \geq q$ and   $\langle e,p \rangle \mapsto \langle e',p'\rangle$ then $\ldash{p'}{q'} e': \tau$.
\end{theorem}
\begin{proof}[Proof notes]
  Proved by nested induction on $\ldash{q}{q'} e : \tau$ and $\langle e,p \rangle \mapsto \langle e',p'\rangle$.
\end{proof}

\paragraph{Alternative soundness theorem:}

Recall the judgement $V \vdash e \Downarrow v \mid (q,q')$
\begin{definition}
$\phi(V : \Gamma) = \sum_{x \in \text{dom}(\Gamma)} \phi(V(x) : \Gamma(x)) $
\end{definition}
where $\Gamma $ assigns types to variables.

\begin{theorem}
Let $V : \Gamma$ and $\Gamma \ldash{q}{q'} e : \tau$ and $V \vdash e \Downarrow v \mid (p,p')$ then $\phi(V:\Gamma) + q \geq p$ and $\phi (V : \Gamma) +  q - \phi (v : \tau) -q' \geq p - p'$
\end{theorem}

This theorem shows that the type derivation is a certificate for bound correctness.


\section{Type inference}

\begin{example}
We want to find a derivation for:
\[\ldash{0}{0} fix\big(id .\ \lambda(x : L(unit)) 
    \ matL(x;\ nil;\ y,ys.cons(y;\ tick\{2\} (id (ys)))\big) : L^2(unit) \rightarrow^{0/0} L^0(unit)\]
\end{example}
  
See figure~\ref{fig:type-inference-example} for the deriviation tree, where $e_{id} = fix\big(id .\ \lambda(x : L(unit)) \ matL(x;\ nil;\ y,ys.cons(y;\ tick\{2\} (id (ys)))\big) $ and $\tau_{id} = L^2(unit) \rightarrow^{0/0} L^0(unit)$.


\begin{sidewaysfigure}[ht]
\begin{prooftree}
\def\defaultHypSeperation{.3em}
\AxiomC{$x : L^2(1) \ldash{0}{0} x : L^2(1)$}
\AxiomC{$id : \tau_{id} \ldash{0}{0} nil : L^0(1)$}
%
\AxiomC{}
\LeftLabel{relax}
\UnaryInfC{$y : 1 \ldash{2}{2} y : 1$}

\AxiomC{}
\LeftLabel{app}
\UnaryInfC{$id : \tau_{id},ys : L^2(1) \ldash{0}{0} id (ys) : L^0(1)$}
\UnaryInfC{$id : \tau_{id}, ys :L^2(1) \ldash{2}{0} tick \{2\} (id (ys)) :  L^0(1)$}

\BinaryInfC{$id : \tau_{id}, y : 1, ys: : L^2(1) \ldash{2}{0} cons(y,tick \{2\} (id(ys)) : L^0(1)$}
%
\TrinaryInfC{$id : \tau_{id}, x : L^2(1) \ldash{}{} e_{id} : L^0(1)$}
\UnaryInfC{$id : \tau_{id} \ldash{0}{0} \lambda(x) e_{id} : \tau_{id}$}
\UnaryInfC{$\ldash{0}{0} fix(id \lambda(x) e_{id} : \tau_{id}$}
\end{prooftree}
\caption{Example of type inference}
\label{fig:type-inference-example}

\end{sidewaysfigure}


For type inference we need algorithmic (or syntax-directed) rules.

\begin{example}
\begin{prooftree}
\AxiomC{$q \geq q'$}
\AxiomC{$\tau <: \tau'$}
\BinaryInfC{$\Gamma, x: \tau \ldash{q}{q'} x : \tau'$}
\end{prooftree}
\end{example}

Algorithm for type inference:
\begin{enumerate}
\item Infer usual tpyes (without annotations), which results in a type derication (like example in figure~\ref{fig:type-inference-example} with all annotations removed);
\item Add potential variables where a potential annotation is required. See figure~\ref{fig:type-inference-example-vars}. Note the this step is only partially done in the figure;
\item Derive from the typing rules linear constraints on potential variables;
\item Solve constraints with LP solver;
\item Objective is the sum of initial potential annotations.
\end{enumerate}

\begin{sidewaysfigure}[ht]
\begin{prooftree}
\def\defaultHypSeperation{.3em}
\AxiomC{$x : L^{p_2}(1) \ldash{r_3}{r'_3} x : L^{p'_2}(1)$}
\AxiomC{$id : \tau_{id} \ldash{}{} nil : L(1)$}
%
\AxiomC{}
\LeftLabel{relax}
\UnaryInfC{$y : 1 \ldash{}{} y : 1$}

\AxiomC{}
\LeftLabel{app}
\UnaryInfC{$id : \tau_{id},ys : L(1) \ldash{s_2}{s'_2} id (ys) : L(1)$}
\UnaryInfC{$id : \tau_{id}, ys :L(1) \ldash{s_1}{s'_1} tick \{2\} (id (ys)) :  L(1)$}

\BinaryInfC{$id : \tau_{id}, y : 1, ys: : L^2(1) \ldash{r_4}{r'_4} cons(y,tick \{2\} (id(ys)) : L(1)$}
%
\TrinaryInfC{$id : \tau_{id}, x : L^{p_1}(1) \ldash{r_3}{r'_3} e_{id} : L^{p'_1}(1)$}
\UnaryInfC{$id : \tau_{id} \ldash{r_1}{r'_1} \lambda(x) e_{id} : \tau_{id}$}
\UnaryInfC{$\ldash{r_0}{r'_0} fix(id \lambda(x) e_{id} : \tau_{id}$}
\end{prooftree}
\caption{Type inference, step 2 of the algorithm}
\label{fig:type-inference-example-vars}
\end{sidewaysfigure}


\section{Implementation in RAML and examples}
Live RAML (Resource aware ML) demo showing binary counter, using the source code displayed in figure~\ref{code:binary-counter}.
\begin{figure}[ht]
\begin{lstlisting}[language=caml,frame=single,basicstyle=\footnotesize]

let rec id x =
    match x with
    | [] -> []
    | y::ys -> y::(let _ = Raml.tick 2.0 in ys)

type bit = Zero | One

let rec inc counter = 
    match counter with
    | [] -> [One]
    | Zero::bs -> One::bs 
    | One::bs -> Zero::(inc bs)

let rec in_many n = 
    match n with
    | Z -> []
    | S n'-> inc (inc_many n')
\end{lstlisting}
\caption{Code for binary counter example}
\label{code:binary-counter}
\end{figure}
Second example with queue.
\end{document}
%%% Local Variables:
%%% mode: latex
%%% TeX-master: t
%%% End:

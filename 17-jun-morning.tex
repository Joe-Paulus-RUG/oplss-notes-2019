%%%%%%%%%%%%%%%%%%%%%%%%%%%%%%%%%%%%%%%%%
% University/School Laboratory Report
% LaTeX Template
% Version 3.1 (25/3/14)
%
% This template has been downloaded from:
% http://www.LaTeXTemplates.com
%
% Original author:
% Linux and Unix Users Group at Virginia Tech Wiki 
% (https://vtluug.org/wiki/Example_LaTeX_chem_lab_report)
%
% License:
% CC BY-NC-SA 3.0 (http://creativecommons.org/licenses/by-nc-sa/3.0/)
%
%%%%%%%%%%%%%%%%%%%%%%%%%%%%%%%%%%%%%%%%%

%----------------------------------------------------------------------------------------
% PACKAGES AND DOCUMENT CONFIGURATIONS
%----------------------------------------------------------------------------------------

\documentclass{article}

\usepackage[version=3]{mhchem} % Package for chemical equation typesetting
\usepackage{siunitx} % Provides the \SI{}{} and \si{} command for typesetting SI units
\usepackage{graphicx} % Required for the inclusion of images
\usepackage{natbib} % Required to change bibliography style to APA
\usepackage{amsmath} % Required for some math elements 



\usepackage{proof}
\usepackage{hyperref}


\setlength\parindent{0pt} % Removes all indentation from paragraphs

\renewcommand{\labelenumi}{\alph{enumi}.} % Make numbering in the enumerate environment by letter rather than number (e.g. section 6)

%\usepackage{times} % Uncomment to use the Times New Roman font

%----------------------------------------------------------------------------------------
% DOCUMENT INFORMATION
%----------------------------------------------------------------------------------------

\title{Session-Typed Concurrent Programming \\ Frank Pfenning \\ OPLSS 2019} % Title

\author{John \textsc{Smith}} % Author name



\begin{document}

\maketitle % Insert the title, author and date

\begin{center}
\begin{tabular}{l r}
Date Performed: & June 17th 2019 \\ % Date the experiment was performed
Students: & J.W.N. Paulus  \\
& R. G. Singh \\ % Partner names
& H. C. A. Tavante \\
Instructor: & Professor Smith % Instructor/supervisor
\end{tabular}
\end{center}

% If you wish to include an abstract, uncomment the lines below
% \begin{abstract}
% Abstract text
% \end{abstract}

%----------------------------------------------------------------------------------------
% SECTION 1
%----------------------------------------------------------------------------------------



\subsection*{Motivation and intro}

There are two ways to look at a program. From an operational standpoint
and form a logical standpoint. Let us look at an example program the
\emph{presumably} gets the smallest element form \emph{presumably} a
list.

\textbf{Operational reasoning}: \emph{How} a program works.

\begin{verbatim}
sort(A)
x = A[0]
\end{verbatim}

Here \texttt{sort} \emph{presumably} sorts the list \texttt{A} inplace.
We assume here that \texttt{A} is a list because we use the
\texttt{{[}{]}} operator on it to \emph{presumably} fetch the first
element of the sorted list.

\textbf{Logical reasoning}: \emph{What} does the program do?\\

\verb|hd |\(\circ\)\verb| sort| \\

Here we use function composition \(\circ\) to capture that we sort the
list and then take the first element. Here, \texttt{hd} is a function
that takes the first element of something, and \texttt{sort} is a
function that sorts its input and returns it. Note that we do not
specify how \texttt{sort} or \texttt{hd} is implemented. And it from a
logical reasoning point of view, we don't care.

\textbf{Designing abstractions}

When we design abstractions we aim to capture some computational
phenomenon such that:

\begin{enumerate}
	\item
	We can reason about it operationally
	\item
	We can reason about it logically
	\item (1) and (2) are coherent
\end{enumerate}

\section{Lecture 1: Sequent Calculus for Singleton logic}

Logic is:

\begin{itemize}
	\item
	The process of \textbf{deduction}
	\item
	About \textbf{inference} rather than the truth
	\item
	Wikipedia: the systematic study of the
	\href{https://en.wikipedia.org/wiki/Logical_form}{form} of
	\href{https://en.wikipedia.org/wiki/Validity_(logic)}{valid}
	\href{https://en.wikipedia.org/wiki/Inference}{inference}, and the
	most general laws of \href{https://en.wikipedia.org/wiki/Truth}{truth}
	\item
	A rapper according to
	\href{https://en.wikipedia.org/wiki/Logic_(musician)}{Wikipedia}
\end{itemize}

\subsection{Form and inference rules}

A sequent has following form: \[
\underbrace{A}_{\text{Antecedent}} \vdash \underbrace{B}_{\text{succedent}}
\] Where the antecedent contains the assumption, and succedent contains
the ``conclusion''. In the case of the \emph{singleton} logic, both
\(A\) and \(B\) must be a singleton, i.e.~exactly one antecedent and one succedent (and they should both be propositions). The \(\vdash\) symbol should be read as ``entails''.

\subsubsection{Connectors}

We use the ``\(\&\)'' for conjunction (and) and the ``\(\oplus\)'' for
disjunction (simple logical or, not to be confused with exclusive or).

\subsubsection{Inference rules}
The fist rule we introduce is the states that we can derive that \(A\) entails \(A\) without needing anything. 
\[
\infer[\textsc{id}_A]{A \vdash A}{}
\]

The second rule we introduce is the \(\textsc{cut}\) rule. It encodes the transitivity of entailment. To derive \(C\) from \(A\) (that is \(A \vdash C\)) we first derive \(A \vdash B\) and \(B \vdash C\) and then we conclude \(A \vdash C\). One could informally say that ``\(B\) is a lemma''.

\[
\infer[\textsc{cut}_B]
{A \vdash C}{A \vdash B & B \vdash C}
\]


\paragraph{Connectors.} So far the propositions in our . Let's introduce some connectors. Each connector will have a "left", introducing.One of the most basic connectors is the "or" connector. We will write it as \(\oplus\) here\footnote{The \(\oplus\) is simple or, and should not be confused with the exclusive or operator}.


\section{Cut-Elimination}


The cut rule is not absolutely necessary. It is possible to write proof without using it. Cut-elimination helps us to show that our logic is consistent (our logic is decidable and we can test all the possibilities).

How can we eliminate a cut? For example:




%----------------------------------------------------------------------------------------
% BIBLIOGRAPHY
%----------------------------------------------------------------------------------------

\bibliographystyle{apalike}

\bibliography{sample}

%----------------------------------------------------------------------------------------


\end{document}
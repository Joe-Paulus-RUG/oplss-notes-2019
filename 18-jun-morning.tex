%%%%%%%%%%%%%%%%%%%%%%%%%%%%%%%%%%%%%%%%%
% University/School Laboratory Report
% LaTeX Template
% Version 3.1 (25/3/14)
%
% This template has been downloaded from:
% http://www.LaTeXTemplates.com
%
% Original author:
% Linux and Unix Users Group at Virginia Tech Wiki 
% (https://vtluug.org/wiki/Example_LaTeX_chem_lab_report)
%
% License:
% CC BY-NC-SA 3.0 (http://creativecommons.org/licenses/by-nc-sa/3.0/)
%
%%%%%%%%%%%%%%%%%%%%%%%%%%%%%%%%%%%%%%%%%

%----------------------------------------------------------------------------------------
% PACKAGES AND DOCUMENT CONFIGURATIONS
%----------------------------------------------------------------------------------------

\documentclass{article}

\usepackage[version=3]{mhchem} % Package for chemical equation typesetting
\usepackage{siunitx} % Provides the \SI{}{} and \si{} command for typesetting SI units
\usepackage{graphicx} % Required for the inclusion of images
\usepackage{natbib} % Required to change bibliography style to APA
\usepackage{amsmath} % Required for some math elements 



\usepackage{proof}
\setlength{\inferLineSkip}{4pt}

\usepackage{hyperref}
\usepackage{dashrule}

\usepackage{bussproofs}




\setlength\parindent{0pt} % Removes all indentation from paragraphs

\renewcommand{\labelenumi}{\alph{enumi}.} % Make numbering in the enumerate environment by letter rather than number (e.g. section 6)

%\usepackage{times} % Uncomment to use the Times New Roman font

%----------------------------------------------------------------------------------------
% DOCUMENT INFORMATION
%----------------------------------------------------------------------------------------

\title{Session-Typed Concurrent Programming \\ Frank Pfenning \\ OPLSS 2019} % Title

\author{John \textsc{Smith}} % Author name



\begin{document}

\maketitle % Insert the title, author and date

\begin{center}
\begin{tabular}{l r}
Date Performed: & June 18th 2019 \\ % Date the experiment was performed
Students: & J.W.N. Paulus  \\
& R. Gurdeep Singh \\ % Partner names
& H. C. A. Tavante
\end{tabular}
\end{center}

% If you wish to include an abstract, uncomment the lines below
% \begin{abstract}
% Abstract text
% \end{abstract}

%----------------------------------------------------------------------------------------
% SECTION 1
%----------------------------------------------------------------------------------------

\section{A counter}

We shall implement a counter with the following:

\[
\; bin\; =\; \oplus\; \{ b_0: bin, b_1:bin, e:\alpha \}\; 
\]

\[
\; bin\; \vdash\; succ\; bin\; 
\]

\begin{equation}
\begin{split}
succ = caseL( b_0 \Rightarrow R . b_1; \leftrightarrow \\
            | b_1 \Rightarrow R . b_0; succ \\
            | e \Rightarrow R . b_1; R . e; \leftrightarrow )
\end{split}
\end{equation}

\begin{equation}
\begin{split}
ctr = \& \{ inc:ctr, reset:ctr, val:bin \}
\end{split}
\end{equation}

\[
\; bin\; \vdash\; counter:ctr\; 
\]


\begin{equation}
\begin{split}
counter = caseR( inc \Rightarrow succ \overset{bin}{\vert} counter \\
              | reset \Rightarrow delete \overset{bin}{\vert} counter \\
              | val \Rightarrow \leftrightarrow )
\end{split}
\end{equation}

For delete:

\[
\; bin\; \vdash\; delete:bin\; 
\]

\begin{equation}
\begin{split}
delete = caseL( b_0 \Rightarrow delete \\
              | b_1 \Rightarrow delete \\
              | e \Rightarrow R . e; \leftrightarrow )
\end{split}
\end{equation}

%----------------------------------------------------------------------------------------
% BIBLIOGRAPHY
%----------------------------------------------------------------------------------------

\bibliographystyle{apalike}

\bibliography{sample}

%----------------------------------------------------------------------------------------


\end{document}